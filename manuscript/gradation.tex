%%% -*-LaTeX-*-

\chapter{A Complexity Spectrum for SweetPea designs}

The complexity of SweetPea designs dramatically varies depending on the user's requirements. At one end of the complexity spectrum are simple designs in which there are no structured relationships between trials in a sequence. At the other end are the most intricate designs, in which many structured relationships exist between multiple trials in a sequence at varying intervals. This chapter defines a precise gradation of five complexity tiers for experimental designs in order to facilitate later discussion, including evaluating different sampling strategies based on the complexity tiers to which they apply.

Recall that a block defines an experimental design. Blocks have three characteristics: a set of factors, a crossing of some subset of the factors, and a set of constraints. This section considers the complexity of a design based solely upon the first characteristic: the factors present. This gradation ignores constraints on the overall design, as their complexity depends primarily upon the complexity of the factors that they constrain.

\section{Complexity Gradation}

The complexity tier for an arbitrary experiment is determined solely by the types of factors and windows present in the design. We will begin with the most straightforward designs and finish with the most complex.

\subsection{Tier 1}

Tier 1 designs consist of basic factors only. No derived factors are allowed, and the crossing contains all factors in the design. Every valid trial sequence is a permutation of the crossed factors.

\begin{verbatim}
direction = Factor("Direction", ["North", "South", "East", "West"])
time_of_day = Factor("Time of Day", ["Sunrise", "High Noon", "Sunset"])
\end{verbatim}

\subsection{Tier 2}

Tier 2 designs still consist exclusively of basic factors, but allow additional basic factors to be present in the design which are uncrossed. Every valid trial sequence is a permutation of the crossed factors paired with random level selections for the uncrossed factors. (The crossing does not bear any impact on the uncrossed factors.)

\subsection{Tier 3}

Tier 3 designs add derived factors with windows being limited to those with $width=1$ and $stride=1$. (\texttt{WithinTrial}) Every valid trial sequence is a permutation of the crossed factors paired with random level selections for the uncrossed basic factors. The values for uncrossed derived factors depend solely upon the level selections of basic factors.

\begin{verbatim}
congruency = Factor("Congruency", [
  DerivedLevel("Congruent",   WithinTrial(op.eq, [color, text])),
  DerivedLevel("Incongruent", WithinTrial(op.ne, [color, text]))
])
\end{verbatim}

\subsection{Tier 4}

Tier 4 designs relax the limitations on derived factors to allow windows with $width=2$ and $stride=1$ as well. (\texttt{Transition}) Valid trial sequences are still permutations of the crossed factors; however, \texttt{Transition} windows may prohibit some permutations when included in the crossing.

\begin{verbatim}
def repeat(colors):
    return colors[0] == colors[1]

def switch(colors):
    return not repeat(colors

color_repeats = Factor("Color Repeats", [
  DerivedLevel("Repeat",    Transition(repeat, [color])),
  DerivedLevel("No Repeat", Transition(switch, [color]))
])
\end{verbatim}

\subsection{Tier 5}

Tier 5 designs further relax the limitations on derived factors to allow windows with any width and stride, so long as they do not create constraints that are impossible to satisfy.

\begin{verbatim}
def repeat(colors):
    return colors[0] == colors[1]

def switch(colors):
    return not repeat(colors)

unconnected_color_repeats = Factor("Color Repeats (Unconnected)", [
  DerivedLevel("Repeat",    Window(repeat, [color], 2, 2)),
  DerivedLevel("No Repeat", Window(switch, [color], 2, 2))
])
\end{verbatim}

\section{Conclusion}

This complexity spectrum identifies five specific tiers for classifying experimental designs. Each tier allows incrementally more complexity than its predecessor. This spectrum proves useful for reasoning about different designs and the complexities involved with generating conforming trial sequences.
