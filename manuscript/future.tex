%%% -*-LaTeX-*-

\chapter{Future Work}

This thesis has presented an approach for constructing trial sequences for some experimental designs based on a bijection between the natural numbers and viable trial sequences. This approach is beneficial in that sampled sequences are guaranteed to be uniformly distributed, and sample construction is nearly instantaneous. This approach represents a vast improvement over the prior approach of generating solutions using SAT sampling. However, there are many opportunities for improvement and additional research.

The counting and construction methods presented here apply to several tiers of experimental designs, yet they do not account for external constraints in the block. They only ensure that no violations of internal constraints (between basic and derived factors) exist and rely on rejection sampling to discard generated samples that violate external constraints. While rejecting sampling is sufficient in practice for many designs, it is possible to construct designs with external constraints that would make the solution space so sparse as to render rejection sampling impractical. Further research to incorporate constraints in the counting and construction process would be worthwhile. We could likely apply combinatoric techniques for generating permutations with prohibited patterns to improve this process.

Additionally, the existing methods cannot yet construct sequences for designs beyond tier 3, which include derived factors spanning multiple trials. (Derived factors with complex windows.) The approaches presented here could reasonably be extended to apply to tier 4 designs (which allow \texttt{Transition} windows) with small adjustments. Increasing their capability to handle tier 5 designs would also be worthwhile, though significantly more difficult due to potential overlap and staggering between different derived factors. Rejection sampling could likely continue to handle constraints even for these advanced designs if we discovered adequate construction methods.

As an implementation detail, the current construction algorithm is single-threaded, generating all samples serially. However, there is no dependent relationship between samples; therefore, we can fully parallelize the construction algorithm for a linear speedup in the number of available cores.

We could also research other approaches which do not rely wholly upon direct construction. Various hybrid approaches have been considered but not thoroughly researched. As an example, a numeric construction approach could be used to construct a partial trial sequence which could then be completed using a SAT sampler. Such a hybrid approach would make the sample generation more efficient for all designs, but care would need to be taken to preserve uniformity guarantees. A partially constructed sequence may over-prune the remaining search space, leading to biased results.

Lastly, improvement opportunities exist in the SAT encoding itself. The naive encoding used currently is easy to understand and reason about but is wasteful in its representation. It uses far more variables than are theoretically required to represent the data. If we research other types of encoding schemes, we may find an alternative that could improve performance. Such an encoding would still likely need to avoid encoding all permutations as distinct solutions to avoid a combinatoric explosion. A hybrid approach, the inverse of the previous paragraph, would likely perform well so long as the size of the independent support set was kept small: A SAT encoding could be developed to partially construct a sequence, including sampling the most complex aspects such as derived factors and constraints, while the remaining values could be populated using a construction approach. Splitting the complexity challenges this way would reduce the burden on the SAT tooling and potentially simplify the construction algorithm. In the same vein, SweetPea could also begin utilizing the native \texttt{XOR} support offered by CryptoMiniSat, which may also offer performance improvements.
