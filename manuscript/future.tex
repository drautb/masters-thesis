%%% -*-LaTeX-*-

\chapter{Future Work}

This paper has presented an alternative approach to constructing trial sequences for a subset of experimental designs based on a bijection between the natural numbers and viable trial sequences. This approach is beneficial in that sampled sequences are guaranteed to be uniformly distributed, and sequences can be constructed nearly instantaneously. This is an improvement over sampling solutions to the current SAT encoding. However, there are many ways in which it could be improved, and other alternatives that should be researched.

The counting and construction methods presented here may be applied to level 1, 2, and 3 designs as defined in chapter 3, yet they do not account for external constraints that may be specified as part of the design. They only ensure that no internal constraints (between basic and derived factors) are violated. While rejecting sampling is sufficient in practice for some designs, one can construct design constraints in which valid solutions are so sparse as to render rejection sampling insufficient. Further research to incorporate constraints in the counting and construction process would be worthwhile. Combinatoric techniques for generating permutations with prohibited patterns would be likely be applicable to this process.

Additionally, the existing methods cannot yet construct sequences for designs beyond level 3 which include derived factors spanning multiple trials. (Complex windows) The approaches presented here could reasonably be extended to apply to level 4 designs (which allow \texttt{Transition} windows) with small adjustments. Increasing their capability to hand level 5 designs would also be worthwhile, though significantly more difficult due to potential overlap and staggering between different derived factors. It likely that rejection sampling could continue to handle constraints even for these advanced designs, if the construction methods could be discovered.

There are other approaches that could also be considered which do not rely wholly upon direct construction. Various hybrid approaches have been considered, but not thoroughly researched. For example, a numeric construction approach could be used to construct a partial trial sequence, which could then be completed using a SAT sampler, once the solution space had been sufficiently reduced. Difficulties may encountered in preserving uniforminty guarantees in this approach, as a partially constructed sequence may overprune the remaning search space, leading to biased results.

Lastly, the SAT encoding itself may be altered. The naive encoding used currently is easy to understand and reason about, but is obviously inefficient. If we had more experience with other types of encoding schemes, an alternative may be discovered that could perform better. Such an encoding would still likely need to avoid encoding all permutations as distinct solutions to avoid the explosion. A hybrid approach, the inverse of the previous paragraph, would likely perform well. A SAT encoding could be developed to partially construct a sequence, including sampling the most complex aspects such as derived factors and constraints, while the remaining, simpler, values could be populated using a construction approach, thus reducing the burden on the SAT tooling.


