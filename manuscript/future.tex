%%% -*-LaTeX-*-

\chapter{Future Work}

This paper has presented an approach for constructing trial sequences for some experimental designs based on a bijection between the natural numbers and viable trial sequences. This approach is beneficial in that sampled sequences are guaranteed to be uniformly distributed, and sequences can be constructed nearly instantaneously. This is a vast improvement over  the prior approach of sampling solutions using SAT Sampling. However, there are many ways in which it could be improved and other alternatives that should be researched.

The counting and construction methods presented here apply to several tiers of experimental designs, yet they do not account for external constraints that may be specified as part of the design. They only ensure that no internal constraints (between basic and derived factors) are violated and rely on rejection sampling to discard generated samples that violate external contraints. While rejecting sampling is sufficient in practice for many designs, it is possible to construct designs with external constraints that would make the solution space so sparse as to render rejection sampling impractical. Further research to incorporate constraints in the counting and construction process would be worthwhile. Combinatoric techniques for generating permutations with prohibited patterns would be likely be applicable to this process.

Additionally, the existing methods cannot yet construct sequences for designs beyond tier 3 which include derived factors spanning multiple trials. (Derived factors with complex windows.) The approaches presented here could reasonably be extended to apply to tier 4 designs (which allow \texttt{Transition} windows) with small adjustments. Increasing their capability to handle tier 5 designs would also be worthwhile, though significantly more difficult due to potential overlap and staggering between different derived factors. It is likely that rejection sampling could continue to handle constraints even for these advanced designs, if the construction methods could be discovered.

As an implementation detail, the current construction algorithm is single threaded, requiring all samples to be generated serially. However there is no dependent relationship between samples, therefore the construction algorithm can be fully parallelized for a linear speedup in the number of available cores.

There are other approaches that could also be considered which do not rely wholly upon direct construction. Various hybrid approaches have been considered, but not thoroughly researched. As an example, a numeric construction approach could be used to construct a partial trial sequence which could then be completed using a SAT sampler, once the solution space had been sufficiently reduced. This would more efficient sample generation for all designs, but care would need to be taken to preserve uniformity guarantees. It is possible that a partially constructed sequence may overprune the remaning search space, leading to biased results.

Lastly, the SAT encoding itself may be altered. The naive encoding used currently is easy to understand and reason about, but is wasteful in its representation. Far more variables are used then are theoretically required to represent the data. If other types of encoding schemes were researched, an alternative may be discovered that could improve performance. Such an encoding would still likely need to avoid encoding all permutations as distinct solutions to avoid a combinatoric explosion. A hybrid approach, the inverse of the previous paragraph, would likely perform well so long as the size of the independent support set were kept small: A SAT encoding could be developed to partially construct a sequence, including sampling the most complex aspects such as derived factors and constraints, while the remaining simpler values could be populated using a construction approach. This would reduce the burden on the SAT tooling and potentially simplify the construction algorithm. In the same vein, SweetPea could also begin utilizing the native \texttt{XOR} support offered by CryptoMiniSat, which may also offer performance improvements.
