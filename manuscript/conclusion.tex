%%% -*-LaTeX-*-

\chapter{Conclusion}

The current implementation of SweetPea provides a useful standard language for defining experimental designs. However, it has not yet fully achieved its objective to generate uniformly-distributed trial sequences conforming to these designs. This failure is due primarily to the inability of current SAT sampling tools to cope with such large and complex solution spaces. While SAT tools are still useful for quickly generating solutions to in-progress designs, SweetPea has not yet been successful in guaranteeing uniformly-distributed samples for arbitrary designs using these tools.

This thesis has presented background information and benchmarks detailing these shortcomings and has introduced a complexity gradation for discussing experimental designs of varying complexity. It also introduced methods for enumerating arbitrary solutions to tier one, two, and three designs. These methods allow efficient generation of samples which are guaranteed to follow a uniform distribution for these design tiers. As these techniques do not yet account for external design constraints, rejection sampling is applied to ensure that no generated sequences violate these constraints.

While not yet able to achieve SweetPea's objectives for all experimental designs, this approach is hugely beneficial for simple designs. Uniformly-distributed samples can be generated exceptionally quickly, and have the added benefit of being numbered. It also provides a framework and pattern upon which to build for more complicated designs. We expect that these methods can be extended or hybridized to guarantee uniformly distributed samples to all experimental designs.
