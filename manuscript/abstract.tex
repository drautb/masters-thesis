%%% -*-LaTeX-*-
%%% This is the abstract for the thesis.
%%% It is included in the top-level LaTeX file with
%%%
%%%    \preface    {abstract} {Abstract}
%%%
%%% The first argument is the basename of this file, and the
%%% second is the title for this page, which is thus not
%%% included here.
%%%
%%% The text of this file should be about 350 words or less.

SweetPea is a new language providing a means of declaratively describing experimental designs. Because of their combinatorial nature, realistic experimental designs have too many conforming trial sequences to reasonably enumerate all of them for random sampling. As a result, uniformly sampled trial sequences must be generated using alternative means. SweetPea currently does this by encoding the design as a SAT formula, to which a SAT sampler is then applied to create uniformly sampled solutions. Although recent SAT samplers can process formulas with large solution spaces, particular characteristics of experimental design have made this approach unfeasible. We present evidence explaining why SAT samplers in their current state are not viable for this problem and lay the foundation for an alternative encoding, based on a bijection from natural numbers to unique trial sequences. This alternative encoding allows us to generate uniformly-distributed random samples for certain classes of designs in a fraction of the time required by previous methods.